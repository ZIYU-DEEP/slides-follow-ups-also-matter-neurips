\documentclass[10pt, xcolor=x11names,compress]{beamer}
\usepackage{tabulary}
% \usepackage{algorithm}
% \usepackage{algpseudocode}
\usepackage{algorithm,algpseudocode}
% \usepackage[lined]{algorithm2e}
\usepackage{etoolbox}
\usepackage{booktabs}
\usepackage{amssymb}
\usepackage{amsfonts}
\usepackage{amsthm}
\usepackage{bm}
\usepackage{float}
\usepackage{graphicx}
\usepackage{mwe}% for example pictures
\usepackage{siunitx}
\usepackage{hyperref}
\usepackage{biblatex}
\addbibresource{main.bib}
\usecolortheme{spruce}
\useoutertheme{infolines}
\usefonttheme[onlymath]{serif}
\setbeamertemplate{headline}[default]
\setbeamertemplate{navigation symbols}{}
\mode<beamer>{\setbeamertemplate{blocks}[rounded][shadow=true]}
\setbeamercovered{transparent}
\input{macros}

%%%%%%%% Set the warning %%%%%%%%
\usepackage{silence}
\WarningFilter[pdftoc]{hyperref}{Token not allowed in a PDF string}
\WarningsOff

%%%%%%%% Set the color %%%%%%%%
\setbeamercolor{block body}{use=structure, fg=white, bg=black!20}
\setbeamercolor{itemize item}{fg=black}
\setbeamercolor{itemize subitem}{fg=gray} 
\setbeamercolor{itemize subsubitem}{fg=black!20} 

%%%%%%%% Set the high-level colorbox %%%%%%%%
\makeatletter\setbeamertemplate{footline}
{  
\leavevmode%  
\hbox{%  

\begin{beamercolorbox}[wd=.5\paperwidth,ht=2.5ex,dp=1ex,center]{author in head/foot}%    
\usebeamerfont{author in head/foot}
\insertshortauthor%~~\beamer@ifempty{\insertshortinstitute}{}
\end{beamercolorbox}%  

\begin{beamercolorbox}[wd=.5\paperwidth,ht=2.5ex,dp=1ex,right]{date in head/foot}%    
\usebeamerfont{date in head/foot}\insertshortdate{}\hspace*{2em}    
\insertframenumber{} / \inserttotalframenumber\hspace*{2ex}   
\end{beamercolorbox}}%  
\vskip0pt%
}

%%%%%%%% Set the first page %%%%%%%%
\makeatother 
\useoutertheme[footline=empty, subsection=false]{miniframes}
\usepackage{multicol}  
\author[C. Wang, Z. Ye, Z. Feng, A. Badanidiyuru, H. Xu]{Chaoqi Wang\inst{1}, Ziyu Ye\inst{1}, Zhe Feng\inst{2}\\Ashwinkumar Badanidiyuru\inst{3}, Haifeng Xu\inst{1}}
\institute[The University of Chicago]{The University of Chicago\inst{1}\vspace{+2pt}\\Google Research\inst{2}\vspace{+2pt}\\Google\inst{3}}
\title{Follow-ups Also Matter:\\Improving Contextual Bandits via Post-serving Contexts}
\date{NeurIPS 2023} 


\begin{document}

\begin{frame}
\titlepage
\end{frame}

% %%%%%%%%%%%%%%%%%%%%%%%%%%%% 
\section{Introduction}
% =======================================
\begin{frame}{Backgrounds}

\begin{itemize}
    \item To give an illustrative example on why post-serving contexts is a practical issue.
    \item To briefly mention that solely dependent on $\vx$ can bring linear regret.
    \item To highlight the research question.
\end{itemize}

\end{frame}

\begin{frame}{Contributions}
\begin{itemize}
    \item \textbf{New framework with practical significance}: 
    \begin{itemize}
        \item Introduced a new family of contextual bandits with post-serving contexts.
    \end{itemize}
    \item \textbf{Enhanced lemma}: 
    \begin{itemize}
        \item Developed a generalized version of the Elliptical Potential Lemma (EPL).
    \end{itemize}
    \item \textbf{Algorithm with theoretical guarantee}: 
    \begin{itemize}
        \item Designed a new algorithm, \texttt{poLinUCB}, enjoying a regret bound of \( \widetilde{\mathcal{O}}(T^{1-\alpha}d_u^{\alpha} + d_u\sqrt{T K })\) when learning with post-serving contexts.
    \end{itemize}
    \item \textbf{Empirical validation}: 
    \begin{itemize}
        \item Achieved SOTA performance on synthetic and real-world datasets.
    \end{itemize}
\end{itemize}


\end{frame}
% =======================================




% %%%%%%%%%%%%%%%%%%%%%%%%%%%%%%%%%%%%%%%
\section{Preliminaries}
% %%%%%%%%%%%%%%%%%%%%%%%%%%%%%%%%%%%%%%%
% =======================================
\begin{frame}[label=Background]{Problem Setup and Notations}

\begin{itemize}
    \item \textbf{Problem Setup}: Each time $t = 1, 2, \cdots, T$:
    \begin{itemize}
        \item The learner observes the context $\vx_t$.
        \item The learner selects an arm $a_t \in [K] $.
        \item The learner observes the reward $r_{t,a_t}$ \textbf{and the post-serving context $\vz_t$}.
    \end{itemize}
\end{itemize}

\begin{itemize}
    \item \textbf{Notations}:
    \begin{itemize}
        \item Actions space: $\calA = [K]$.
        
        \item Pre-serving context: $\vx \in \sR^{d_{\vx}}$; post-serving context: $\vz \in \sR^{d_{\vz}}$.
        \begin{itemize}
            \item $\vz = \phi^{\star}\left(\vx_t\right)+\vepsilon_t, \text{and } \phi^{\star}(\vx)=\sE[\vz \mid \vx]$
        \end{itemize}
        
        \item Reward function: 
        \begin{itemize}
            \item $r_{a}(\vx, \vz) = \vx^\top \vtheta_a^\star + \vz^\top \vbeta_a^\star + \eta$, where $\eta$ is $R_{\eta}$-sub-Gaussian.
        \end{itemize}
        
        \item Matrix representation: 
        \begin{itemize}
            \item $\mX_{t} = \sum_{s=1}^t \vx_s\vx_s^\top + \lambda \mI$ and $\mZ_t = \sum_{s=1}^t \vz_s \vz_s^\top + \lambda \mI$.
        \end{itemize}
        
        \item Norm restrictions: 
        \begin{itemize}
            \item $\forall a \in \calA, \|\vtheta_a^\star\|_2\leq 1, \|\vbeta_{a}^\star\|_2\leq 1 $; $\|\vx\|_2\leq L_x, \|\vz\|_2 \leq L_z$.
        \end{itemize}
        
    \end{itemize}
\end{itemize}

\end{frame}
% =======================================

% =======================================


\begin{frame}{Assumption: Generalized learnability of $\phi^*$}

TODO: make this more abstract and easily digested by the audience.

There exists an algorithm that, given $t$ pairs of examples $\{(\vx_s, \vz_s)\}_{s=1}^t$ with  arbitrarily chosen $\vx_s$'s,   outputs an estimated function of $\phi^\star: \mathbb{R}^{d_x} \rightarrow \mathbb{R}^{d_z}$ such that for any $\vx\in \mathbb{R}^{d_x}$,  the following holds with probability at least $1-\delta$, 

\begin{align}
     e_t^\delta:=\left\|\widehat{\phi}_t(\boldsymbol{x})-\phi^{\star}(\boldsymbol{x})\right\|_2 \leq C_0 \cdot\left(\|\boldsymbol{x}\|_{\boldsymbol{X}_t^{-1}}^2\right)^\alpha \cdot \log (t / \delta), \nonumber
\end{align}
where $\alpha \in (0, 1/2]$ and $C_0$ is some universal constant. 

\end{frame}
% =======================================


% %%%%%%%%%%%%%%%%%%%%%%%%%%%%%%%%%%%%%%%
\section{Method}
% %%%%%%%%%%%%%%%%%%%%%%%%%%%%%%%%%%%%%%%

% =======================================
\begin{frame}[label=warmup]{Warmup: Why Natural Attempts May be Inadequate?}

\end{frame}
% =======================================

% =======================================
\begin{frame}{Generalized Elliptical Potential Lemma}



\end{frame}
% =======================================

% =======================================
\begin{frame}{The Proposed Algorithm: \polinucb}
\hspace{+10pt}
\begin{center}

\scalebox{0.8}{\begin{minipage}[c][0.3\paperheight][c]{1\textwidth}
  \centering

\begin{algorithm}[H]
\caption{\polinucb~(Linear UCB with post-serving contexts)}\label{alg:polinucb}
\begin{algorithmic}[1]
    % \textbf{Input:} Text Document.\newline
    % \textbf{Output:} Summary sentences.
    \State{Initialize parameters $\left\{\boldsymbol{A}_{i, 0} \leftarrow \mathbf{I}^d, \boldsymbol{b}_{i, 0} \leftarrow \mathbf{0}^d\right\}$ for all $i \in[K]$ and $\alpha \in \mathbb{R}^{+}$.}
    \For{$t=0, 1, \ldots, T$}
        \State Receive the \emph{pre-serving context} $\vx _{t}$
        \State Compute the optimistic parameters by maximizing the UCB objective $${\left(a_t, {\widetilde{\phi}_{t}(\vx_t)}, \widetilde{\vw}_t\right) = {\underset{(a, \phi, \vw_a) \in [K] \times \calC_{t-1}\left(\widehat{\phi}_{t-1}, \vx_t\right) \times \calC_{t-1}(\widehat{\vw}_{t-1,a})  }{\argmax}} { \begin{bmatrix} \vx_t \\ \phi(\vx_t) \end{bmatrix}^\top \vw_{a} }}.$$ 
        \State Play the arm $a_t$ and receive the realized \emph{post-serving context} as $\vz_t$ and  the real-valued  reward $$r_{t, a_t} =   \begin{bmatrix} \vx_t \\ \vz_t \end{bmatrix}^\top \vw_{a_t}^{\star}  + \eta_t. $$
        \State Compute $\widehat{\vw}_{t,a}$ using Equation~\ref{eqn:w-closed-form} for each $a \in \calA$.
        \State Compute the estimated post-serving context generating function $\widehat{\phi}_t(\cdot)$ using ERM. 
        \State Update confidence sets $\calC_t(\widehat{\vw}_{t, a})$ and $\calC_t(\widehat{\phi}_t, \vx_t)$ for each $a$ based on Equations~\ref{eqn:w-confidence-set} and \ref{eqn:phi-confidence-set}.
    \EndFor

\end{algorithmic}
\end{algorithm}
\end{minipage}
}
\end{center}


% \begin{center}
% \scalebox{.8}{                      %new code
%     \begin{algorithm}[H]              %new code
%         \DontPrintSemicolon
%         \KwIn{Labels: set of all labels}
%         \For{each XRow $\in$ XSet}{
%         \For{each LRow $\in$ LSet}{
%         CandidateSet $\gets$ $\phi$\;
%         \For{each $label \in Labels$} {
%         $C \gets \phi$\;
%         \For{each $l_i \in LRow$} {
%         \If{\underline{Lat.conflictsWith}($l_i$,label)}{
%         $C \gets C \cup \{x_i\}$ \;
%         }
%         }
%         \If{!\underline{isAffectedBy}(F,$X_{row}$,C)}{
%         CandidateSet $\gets$ CandidateSet $\cap$ \{label\} \;
%         }
%         }
%         $SHF_{row} \gets$ Lat.ChooseMin(CandidateSet) \;
%         Output $X_{row}$, $T_{row}$, $SHF_{row}$ \;
%         }
%         }
%     \end{algorithm}
% }
% \end{center}
\end{frame}
% =======================================


% %%%%%%%%%%%%%%%%%%%%%%%%%%%%%%%%%%%%%%%
\section{Results}
% %%%%%%%%%%%%%%%%%%%%%%%%%%%%%%%%%%%%%%%
% =======================================
\begin{frame}{Regret Analysis}

\begin{table}[]
\centering
\begin{tabular}{@{}ll@{}}
\toprule
\textbf{Settings}  & \textbf{Ours} \\ \midrule
Ours               & $\widetilde{\mathcal{O}}\left(T^{1-\alpha}d_u^{\alpha} + d_u\sqrt{T K }\right)$           \\
Action-dependent contexts   & $\widetilde{\mathcal{O}}\left(T^{1-\alpha}d_u^{\alpha}\sqrt{K} + d_u\sqrt{T K }\right)$            \\
Same setting as in [Abbasi et al., 2011]~\footfullcite{abbasi2011improved} & $\widetilde{\mathcal{O}}\left(T^{1-\alpha}d_u^{\alpha} + d_u\sqrt{T  }\right)$            \\ \bottomrule
\end{tabular}
\caption{Upper bound of regret of \polinucb.}
\label{tab:regret}
\end{table}
\end{frame}
% =======================================

% =======================================
\begin{frame}{Experimental Results}

\begin{itemize}
    \item  Our proposed \polinucb consistently outperforms other strategies\\\textcolor{gray}{{\footnotesize (except for \texttt{LinUCB} which has access to the post-serving context during arm selection)}}.
\end{itemize}

\begin{figure}[h]
    \centering
    \includegraphics[width=0.85\textwidth]{figs/synthetic-comparisons.pdf}
    \vspace{0.2cm}
    \caption{{\small Cumulative Regret in three synthetic environments. The shaded area denotes the standard error computed using 10 different random seeds.}}
    \label{fig:synthetic-experiments}
\end{figure}

\end{frame}
% =======================================

% =======================================
\begin{frame}
 \begin{center}
		{\Huge Thank you.}
		%\bigskip\bigskip % Vertical whitespace
	\end{center}
\end{frame}
% =======================================

\end{document}